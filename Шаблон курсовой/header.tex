%!TEX TS-program = xelatex
\documentclass[a4paper]{article} %XeLateX

%подключаем кириллицу

\usepackage[utf8]{inputenc} 
%\usepackage[cp1251]{inputenc}
\usepackage[T1, T2A]{fontenc}

%%% Работа с русским языком
\usepackage[english, russian]{babel}   %% загружает пакет многоязыковой вёрстки
\usepackage{fontspec}      %% подготавливает загрузку шрифтов Open Type, True Type и др.
\defaultfontfeatures{Ligatures={TeX},Renderer=Basic}  %% свойства шрифтов по умолчанию
\setmainfont[Ligatures={TeX,Historic}]{Times New Roman}


\usepackage[14pt]{extsizes} % 14 шрифт
\usepackage[onehalfspacing]{setspace} % полуторный интервал
\usepackage{indentfirst} %красная строка в первом абзаце

\usepackage[left=3.5cm,top=2cm,bottom=2cm,right=2.0cm]{geometry} %настройка полей

\usepackage{amsmath,amssymb,xcolor,hyperref,graphics, tikz, pgfplots, qtree,csquotes} %Пакеты для формул, изображений, деревьев и т.д.
\usetikzlibrary{positioning,trees,arrows, calc} %Библиотеки для tikz
\usepackage{sgamevar}
% Пакет для настройки оглавления
\usepackage{tocloft} 



\renewcommand{\cftsecfont}{\hspace{0pt}} 
% Имена секций в содержании не жирным шрифтом
\renewcommand\cftsecleader{\cftdotfill{\cftdotsep}}
% Точки для секций в содержании
\renewcommand\cftsecpagefont{\mdseries}  
%Номера страниц не жирные
\setcounter{tocdepth}{3}  
%Глубина оглавления, до subsubsection


\addto \captionsrussian{\renewcommand{\refname}{\center{Список использованной литературы}}} %Список использованной литературы
\addto \captionsrussian {\renewcommand{\contentsname}{Оглавление}} %Оглавление

\makeatletter \renewcommand{\@biblabel}[1]{#1.} \makeatother   % убрать скобки в нумерации библиографии
\makeatletter \renewcommand{\@seccntformat}[1]{\csname the#1\endcsname.~} \makeatother  %добавить точки после номера главы

%%%%%%%%%%%%%%%%%%%%%%%%%%
%Настройки библиографии
%%%%%%%%%%%%%%%%%%%%%%%%%%

\usepackage[
%style=gost-authoryear-min,
style=gost-authoryear,
mergedate=maximum, %перемещает дату в начал
movenames=false, % не перемещает имена авторов после /
maxbibnames=4, % максимальное число авторово
%style=gost-numeric, 
%style=gost-footnote-min,
%style=authoryear,
%style=verbose,
%style=authoryear-icomp,
%style=philosophy-modern,
%style=mla,
%style=apa,
%style=bwl-FU,
%style=nature,
%style=chem-acs,
%style=ieee,
firstinits=true,
isbn=false,
url=false,
%doi=false,
dashed = false,
autolang=other,
%natbib=true,
defernumbers=true
]{biblatex}

%убирает пробел перед : и ;
\renewcommand*{\addcolondelim}{\addcolon\space}
\renewcommand*{\addsemicolondelim}{\addsemicolon\space}

%%Меняем [] на ()%%%%%%%%%%
\renewcommand{\mkbibbrackets}[1]{\mkbibparens{#1}}

%Если нет langid, то считается английский
\DeclareSourcemap{
  \maps[datatype=bibtex]{
    \map{
      \step[fieldset=langid, fieldvalue={english}]
    }
  }
}


%%% Русские источники в начале
%%% Русские источники в начале
\DeclareSourcemap{
\maps[datatype=bibtex]{
\map{
\step[fieldsource=langid, match=russian, final]
\step[fieldset=presort, fieldvalue={a}]
}
\map{
\step[fieldsource=langid, notmatch=russian, final]
\step[fieldset=presort, fieldvalue={z}]
}
}
}

\AtEveryBibitem{%
\clearfield{month}%
\ifentrytype{book}{%
    \clearfield{url}%
    \clearfield{urldate}%
    \clearfield{review}%
    \clearfield{series}%
    \clearfield{pages}}}

%%%%Настройки ГОСТА
\renewcommand{\newblockpunct}{%  
\addperiod\space\bibsentence}%убирает —%block punct.,\bibsentence is for vol,etc.

\DefineBibliographyStrings{english}{%
  pages = {P\adddot},}
\DefineBibliographyStrings{english}{%
  number = {No\adddot},}
\DefineBibliographyStrings{russian}{%
  pages = {c\adddot},}  

%\DeclareLanguageMapping{english}{english-apa} % убирает yearsmonthday в apa
%\DeclareLanguageMapping{russian}{russian-apa}

%убирает пробел перед : и ;
%\renewcommand*{\addcolondelim}{\addcolon\space}
%\renewcommand*{\addsemicolondelim}{\addsemicolon\space}

%%%%%%%%%%%%%%%%%%%%%%%%%%%%%%%%%%%%%%



